\begin{table}[ht]
  \centering
  \caption[Example of deviation of the trajectory of a particle in an IPM]
  {Example of deviation of the trajectory of a particle in an IPM. A particle is released in the center of an IPM with a straight field everywhere, but in a certain range a parasitic field component is added and set to $1\,\mathrm{\%}$ of the main field. The 0 coordinate is the IPM center whereas the readout is located at distance of \(5 \, \mathrm{cm}\).}
  \label{chap3:Deviation}
  \begin{tabular}{ccccccc}
    \toprule
                                   &           & \multicolumn{5}{c}{Range (\(\mathrm{cm}\))}                                                 \\
    \cmidrule(lr){3-7}
                                   & \([0,5]\) & \([0,1]\)                                   & \([1,2]\) & \([2,3]\) & \([3,4]\) & \([4,5]\) \\
    \midrule
    Deviation (\(\mathrm{\mu m}\)) & \(500\)   & \(347\)                                     & \(85\)    & \(42\)    & \(20\)    & \(5.6\)   \\
    \%                             & $100$     & $69.4$                                      & $17$      & $8.4$     & $4$       & $1.12$    \\
    \bottomrule
  \end{tabular}
\end{table}
