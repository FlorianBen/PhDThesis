\begin{table}[ht]
	\centering
	\caption[Example of deviation of the trajectory of a particle in an IPM]
	{Example of deviation of the trajectory of a particle in an IPM. A particle is released in the center of an IPM with a straight field everywhere, but in a certain range a parasitic field component is added and set to $1\,\mathrm{\%}$ of the main field. The 0 coordinate is the IPM center whereas the readout is at \(5 \, \mathrm{cm}\) distance.}
	\label{chap3:Deviation}
	\begin{tabular}{cccccc}
		\toprule
		                               & \multicolumn{5}{c}{Range (\(\mathrm{cm}\))}                                                 \\
		\cmidrule(lr){2-6}
		                               & \([0,1]\)                                   & \([1,2]\) & \([2,3]\) & \([3,4]\) & \([4,5]\) \\
		\midrule
		Deviation (\(\mathrm{\mu m}\)) & \(347\)                                     & \(85\)    & \(42\)    & \(20\)    & \(5.6\)   \\
		\bottomrule
	\end{tabular}
\end{table}