\begin{figure}[!ht]
  \begin{center}
    \begin{tikzpicture}[scale=1,transform shape,american voltages]
      %\ctikzset{bipoles/length=1cm}
      \draw (0,0) node[op amp] (opamp) {} 
      (opamp.-) to [short, i=$i_s$] ($(opamp.-)-(2,0)$) to [R=$R_s$] ($(opamp.-)-(3,0)$) -| ($(opamp.+)-(3.5,.5)$)
      %($(opamp.-)-(2,0)$) node[left]{$V_{in}$} to [short, i=$i_s$] (opamp.-)
      %(0,0) node[op amp]
      
      (opamp.-) |- ($(opamp.-)+(0.2,1)$) to[R=$R_f$] ($(opamp.-)+(2.2,1)$) -|
      (opamp.out)
      (opamp.-) |- ($(opamp.-)+(0.2,2.5)$) to[C=$C_{f}$] ($(opamp.-)+(2.2,2.5)$) -|
      (opamp.out)
      to[short,*-] ($(opamp.out)+(.5,0)$) node [right] {$V_{out}$} node [ocirc] {} 
      (opamp.+) to[short]  ($(opamp.+)-(0,.5)$) node[ground] {};

    \end{tikzpicture}
  \end{center}
  \caption[Typical circuit of a charge amplifier with an operational amplifier]{Typical circuit of a charge amplifier with an operational amplifier.}
  \label{chap3:AOPcharge}
\end{figure}
