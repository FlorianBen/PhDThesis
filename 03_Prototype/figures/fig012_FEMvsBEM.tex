\begin{figure}[!h]
	\begin{subfigure}{.5\textwidth}
		\centering
		\includegraphics[width=\textwidth]{03_Prototype/figures/fig012_BEMa.png}
		\caption{Configuration 1 solved with BEM.}
		\label{}
	\end{subfigure}\hfill
	\begin{subfigure}{.5\textwidth}
		\centering
		\includegraphics[width=\textwidth]{03_Prototype/figures/fig012_FEMa.png}
		\caption{Configuration 1 solved with FEM.}
		\label{}
	\end{subfigure}
	\vskip\baselineskip
	\begin{subfigure}{.5\textwidth}
		\centering
		\includegraphics[width=\textwidth]{03_Prototype/figures/fig012_BEMb.png}
		\caption{Configuration 2 solved with BEM.}
		\label{}
	\end{subfigure}\hfill
	\begin{subfigure}{.5\textwidth}
		\centering
		\includegraphics[width=\textwidth]{03_Prototype/figures/fig012_FEMb.png}
		\caption{Configuration 2 solved with FEM.}
		\label{}
	\end{subfigure}
	\caption[Comparison beetwen BEM and FEM for two different IPM configurations]{Comparison beetwen BEM and FEM for two different IPM configurations. The configuration 1 covers the case of simple parallel plates. In configuration 2, the field is constrained by field correctors on each side (see Section \ref{chap3:field_corrections}). Some differences appear for configuration 2 especially on the side of the IPM where the field variations are important.}
	\label{chap3:FEMvsBEM}
\end{figure}
