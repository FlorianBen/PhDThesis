\begin{figure}[!ht]
	\centering
	\begin{subfigure}[t]{0.7\textwidth}
		\includesvg[width=\textwidth]{03_Prototype/figures/fig022_2IPM_ASYM_NODISK_NODEG_a}
		\caption[Electric field without disk]{Electric field without disk. The minimal value of the longitudinal distortion (in yellow) is not in the IPM center.}
		\label{chap3:no_disk}
	\end{subfigure}

	\begin{subfigure}[t]{0.7\textwidth}
		\centering
		\includesvg[width=\textwidth]{03_Prototype/figures/fig022_2IPM_ASYM_DISK_NODEG_c}
		\caption[Electric field with disks]{Electric field with disks.The disks are inserted to enclosed the two IPMs. The longitudinal distortion (in yellow) is now close to the middle of IPMs.}
		\label{chap3:disks}
	\end{subfigure}
	\caption[Influence of shielding disks on the IPM electric field]{Influence of shielding disks on the IPM electric field along the LWU. Each curve represents an electric field component normalized by the expected perfect field value, within the IPM and along the longitudinal direction. In an ideal IPM the field in the extraction direction should be maximum whereas the two others should be null. More plots are available in Appendix \ref{anx:COMSOL}.}
	\label{chap3:IPM_disk}
\end{figure}
