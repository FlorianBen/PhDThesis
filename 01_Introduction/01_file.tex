\chapter{Brief introduction to neutron science}
\chaptermark{Brief introduction to neutron science}
\cleardoublepage

\minitoc

\section{Introduction}
\begin{refsection}
  \label{ch1:Introduction}
  This chapter presents the general context behind the European Spallation Source: the neutron science. In facts, this thesis has almost nothing related to neutron science and this chapter only serves as a quick overview of the different themes in neutron science in order to answer the following question: Why building an 1.8 billion € neutron source is crucial for sciences? 
 
  After a brief historical review about neutron, this chapter presents what a is neutron and why its properties are interesting for probing structures and matter. Concrete applications of neutron sciences including usage for different domains will be illustrated. Finally, the different ways of producing a neutron are also briefly presented as well as the advantages and drawbacks of each method.

	\section{History}

	\section{What is a neutron?}
	\label{ch1:sec:Neutron}
	Test2 \cite{osti_656719}


	\section{Neutronic science, applications and perspective}

  
  \section{Neutron production}
  \subsection{Composite sources}
  \subsection{Nuclear reactors}
  \subsection{Spallation sources}

	\section{The need of a European Spallation Source}
	\label{ch1:Summary}

	\cleardoublepage
	\section{Bibliography}
  
	\printbibliography[heading=subbibliography]
\end{refsection}