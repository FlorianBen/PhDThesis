\chapter{Brief introduction to sciences using neutrons}
\chaptermark{Brief introduction to sciences using neutrons}
\cleardoublepage

\minitoc

\section{Introduction}
\begin{refsection}
  \label{ch1:Introduction}
  This chapter presents the general context behind the European Spallation Source. In facts, this thesis has almost nothing related to neutron and this chapter only serves as a quick overview of the different themes in science using neutrons in order to answer the following question: Why building an 1.8 billion € neutron source is crucial for Sciences in Europe?

  After a brief historical review about the neutron discovery, this chapter presents what is neutron and why its properties are interesting for probing matter and structures. A concrete application of neutron sciences will be illustrated. Finally, the different ways of producing a neutron are also briefly presented as well as the advantages and drawbacks of each method.

  \section{History}
  Historically, the neutron was the last component of the atom to be discovered. The electron was discovered first by J.J. Thomson in 1897 using a vacuum tube. The electrons are elementary negatively charged components of all atoms. Since the atoms are neutral there must be an other component of the atom with an opposite charge. However, Thomson was not able to really answer how the atom is structured.

  Almost in parallel, the discovery of natural radioactivity by H. Becquerel in 1896 and M. and P. Curie in 1898 opened a new field of possibility for physical experiments. In 1909, E. Rutherford observed that atoms consist of a nucleus that concentrates all mass and charges \cite{Rutherford:1911zz}. 
  From this observation he proposed a new atomic model, later improved by N. Bohr. Then in 1919, he observed that light atoms eject hydrogen when they are impinged by alpha particles: the proton was discovered. However, theses experiment does not explain the differences mass and charge between atoms.

  In 1930, W. Bothe and H. Becker find out that neutral radiations were emitted when beryllium, boron or lithium are bombarded by alpha particles (original paper \cite{Bothe1930}). But they were not able to discriminate against the nature of these neutral particles.
  
  This has been done by I. and F. Joliot-Curie by means of an ionization chamber. They observed that the neutral radiation produces protons when it impacts a light hydrogen-based compound. However, they concluded that the proton ejection could be due to a photon scattering (similar to Compton scattering which was discovered 8 years before) \cite{curie:jpa-00233129}.

  J. Chadwick performed a more accurate measurements of proton recoils for several light elements \cite{Chadwick1932}. The hypothesis of an elastic collision with gamma rays cannot be possible according to the law of energy conservation. The neutral particle must have a mass close to that of protons: the neutron was discovered.

  \section{Neutron and neutron interaction with matter}
  \label{ch1:sec:Neutron}
  A neutron is a non-elementary particle with no charge. Its mass is $939.56\,\mathrm{MeV/c^{2}}$ ($1.67 \cdot 10^{-27}\,\mathrm{kg}$). A neutron does not have strong electromagnetic interaction with the electron of atoms, unlike charged particles or photon. Therefore, neutrons can travel long distances without major interaction. The neutron has a spin of $\frac{1}{2}$ and a low magnetic moment. Therefore it is possible to polarize a neutron beam. The magnetic moment allows it to interact with magnetic fields in the structures of matter. A neutron interacts mainly with the elements of the nucleus through strong interactions. A free neutron has a lifetime of $881.5\,\mathrm{s}$ and decays into a proton, an electron, and an antineutrino.
  Neutrons follow the principle of wave-particle duality. They can therefore be studied both as a particle and as a wave. Neutrons can be distinguished by their energies, velocities or wavelengths:

  \begin{equation*}
    E_{neutron} = \frac{h^{2}}{2m_{neutron}\lambda^{2}}
  \end{equation*}

  There are several ranges with more or less arbitrary limits:

  \begin{table}[ht]
  \centering
  \caption[ESS nominal conditions]
  {ESS nominal conditions.}
  \label{chap1:tab:neutronsT}
  \begin{tabularx}{\linewidth}{XX}
    \toprule
    Characteristic        & Energy              \\
    \midrule
    Relativistic neutron  & $> 2\,\mathrm{MeV}$ \\
    Fast neutrons         & $< 2\,\mathrm{MeV}$ \\
    Intermediate neutrons & $..$                 \\
    Epithermal neutrons   & $..$                 \\
    Thermal neutrons      & $..$                 \\
    Cold neutrons         & $..$                 \\
    \bottomrule
  \end{tabularx}
\end{table}

  Fast neutrons can be slow down precisely with the help of moderators. These moderators are mainly composed of light elements. Indeed the energy transfer during elastic scattering on light elements is efficient. The most common moderators are hydrogen, water, heavy water and graphite.

  % TODO: Finir
  There are several processes behind the interaction of neutrons with matter \cite{Leo1994} that can be classified in two categories: collision or absorption processes.
  In a collision the nucleus structure remains the same, the energy and momentum of the nucleus and/or neutron change. The collision can be:
  \begin{itemize}
    \item Elastic: In an elastic collision the kinetic energy of the system is preserved. The neutron transfers part of its kinetic energy to the nucleus. The direction of the neutron is changed.
    \item Inelastic: In an inelastic collision the kinetic energy of the system is not preserved and the direction of the neutron is changed as well.
  \end{itemize}
  In absorption processes, the nucleus structure is altered by the neutron.
  Such processes have been subdivided as follows:
  \begin{itemize}
    \item Radiative capture: The neutron is captured by the nucleus with a gamma emission
    \item Charged capture: The phenomenon similar to the previous one, except that the excited nucleus emits one or more charged particles
    \item Fission: This phenomenon concerns heavy elements which, under the impact of a neutron, separate into two lighter elements
  \end{itemize}
  These three absorption processes occur mainly when the neutron energy is low. There are also other high-energy absorption phenomena. The total cross section is the sum of all previous cross sections, which leads to quite complex cross section spectra. The measurements of neutron cross sections remain a fundamental aspect of nuclear physics and application.

  \section{Application of neutron probes}
  
  \begin{figure}[!ht]
	\begin{center}
		\includegraphics[width=\textwidth]{01_Introduction/figures/fig000_Time_Scale}
	\end{center}
	\caption[]{}
	\label{chap1:fig:Time_Scale}
\end{figure}


  \begin{figure}[!ht]
	\begin{center}
		\includegraphics[width=\textwidth]{01_Introduction/figures/fig000_Dague}
	\end{center}
	\caption[]{\cite{LEHMANN2012S35}}
	\label{chap1:fig:Time_Scale}
\end{figure}



  \section{Neutron production}
  Producing neutrons is not a trivial task compared to producing charged particles or even photons. The following sections present the principle of operation of thermal neutron user facilities. The term user facilities means that the installation must be open for external user from various scientific domains. Facilities dedicated to specific measurements using neutrons (thermal or non-thermal) are not considered.
  %Obviously, the list is not exhaustive and complete review of existing 

  \subsection{Radioisotope sources}
  The first neutron sources were based on radionuclides. Two distinct types of sources can be used. The first one is based of $\alpha$ or $\gamma$ emitters, generally encapsulated in Beryllium, Lithium or Bore media, leading to the this nuclear reaction for instance::
  \begin{equation*}
    _{2}^{4}\alpha + _{4}^{9}Be \rightarrow _{6}^{12}C + _{0}^{1}n
  \end{equation*}
  This type of source is inexpensive but its activity depends on the alpha emitter activity and the quality of the mixture.
  
  The second type of source is based on radioisotopes that undergo spontaneous fission decays. The amount of neutrons created during the fission depends on the radioelement. The most emissive radioelement is the Californium-252 that emits around $2.314 \cdot 10^{6}$ neutrons per second per microgram and has an half life of $2.63$ years.
  $^{252}Cf$ is a synthesized element usually created in a nuclear reactor so it is expensive to produce.

  The main advantage of these sources are their compactness and their "ease of use". They are suitable for small size experiments, but does not scale for larger experiments. For this purpose, they were quickly overtaken by research reactors.

  \subsection{Nuclear reactors}
  An Uranium-235 atom splits into two lighter nuclei under the impact of a neutron. The fission reaction also leads to neutron emissions and energy release:
  \begin{equation*}
    _{92}^{235}U + _{0}^{1}n \rightarrow _{92}^{236}U^* \rightarrow X + Y + k \times _{0}^{1}n
  \end{equation*}
  with the fission fragments X and Y, and k the number of neutrons released during the fission reaction. The neutrons are thermalized in order to increase the probability of further fissions on other Uranium-235 atoms, creating a chain reaction. In average, each fission of Uranium-235 atoms generates around $2.5$ neutrons.

  The fission reaction is the basis of nuclear power plants and research reactors used to produce neutrons.
  Currently, it is the most widely used method to produce steady state intense neutron beam. The moderation of neutrons is done directly in the reactor pool.
  Some reactor can work in pulsed mode, but it is not straightforward, whilst it is more efficient using neutron chopper afterward.

  About ten research reactors dedicated to users are open in Europe, two of them are located in France:
  \begin{itemize}
    \item The Laue-Langevin Institute (\acrshort{ill}): an international facility based on a $58\,\mathrm{MW}$ high flux reactor.
    \item The Léon Brillouin Laboratory (\acrshort{llb}): a national facility based on a $14\,\mathrm{MW}$ high flux reactor.
  \end{itemize}
  The number of users for both these facilities represents one third of the total number for all European neutron facilities.

  Fusion reactions also produce neutrons, but current technology does not allow exploitation as a neutron source.

  \subsection{Spallation sources}
  The term spallation defines the process of neutrons production by bombarding a target with energetic heavy particles (protons, deuterons, neutron). The description of this model was proposed by R. Saber in 1946 \cite{PhysRev.72.1114}. The process takes place in two stages. When the incident particle has sufficient energy, typically between $200\,\mathrm{MeV}$ and a few $\mathrm{GeV}$, it can interact with several nucleons of a nucleus per intranuclear cascade. In this process nucleons are ejected from the nucleus. After a cascade, the nucleus is in an excited state that can lead to several forms of de-excitation: mainly fissions or evaporation of light elements. Fig. \ref{chap1:fig:spallation} illustrates the spallation process.

  The neutrons are generated in a wide spectrum whose maximum energy is slightly below the energy of the incident particle energy. The number of neutrons produced by spallation depends on the properties of the target and the incident particle. By choosing a dense target the probability of interaction can be maximized. A dense material also stops most of the emitted particles, except gammas and neutrons. The most popular materials are Tungsten, Lead and Mercury as well as actinides\footnote{Actinides should be avoided because they lead to unwanted fissions}. The optimal energy to trigger spallation reactions is between $2\,\mathrm{GeV}$ and $5\,\mathrm{GeV}$.

  A spallation neutron source uses accelerator and a target to produce neutrons following the process described above. A spallation source is totally controlled by its accelerator, allowing to produce intense neutron pulse. It is therefore a complementary method to the research reactor that provide only a continuous flux.

  The first studies of this type of source were carried out in the early 1970s and the first generation sources were built in the late of 1970s \cite{klein1994}. In Europe the first major spallation source was ISIS (UK) inaugurated in 1985 \cite{THOMASON201961}. These sources met success and a new generation of spallation sources has been considered. The second generation sources were achieved in 2000-2010 with \acrshort{sns} (ORNL, USA) \cite{Mason2005}, \acrshort{jsns} (J-PARC, Japan) \cite{Ikeda2002} and \acrshort{csns} (China) \cite{Chen2016}. However, no major second-generation source has been built in Europe \footnote{SINQ \cite{WAGNER2006541} (\acrshort{psi}, Switzerland) is not considered since it is a steady state spallation source.}.

  \begin{figure}[!ht]
	\begin{center}
		\includegraphics[width=0.8\textwidth]{01_Introduction/figures/fig000_spallation}
	\end{center}
	\caption[Schematic view of the spallation process]{Schematic view of the spallation process \cite{gorbinet:tel-00660583}. The incident protons interact with nucleons of the target. An intranuclear cascade occurs leaving the target atom in an excited state. Depending on the properties of the excited nucleus, different de-excitation process may occur. IMF stands for intermediate mass fragment.}
	\label{chap1:fig:spallation}
\end{figure}


  \section{The need of a European Spallation Source}
  \label{ch1:Summary}
  Neutron probe is a popular tool in the European scientific community and for a long time Europe was a leader in neutron production. However, the future does not look so good with a decrease of the time access on neutron lines for scientists. These are the conclusions of an European Strategy Forum on Research Infrastructures (\acrshort{esfri}) report in 2016 \cite{neutron2016}.

  The majority of neutron sources is based on reactors built before the 1980s.
  The LLB reactor is 39 years old whereas ILL reactor diverged in 1967. In Europe, despite the success of ISIS\footnote{ISIS was supposed to operate during 20 years but is still running after 35 years.}, no second generation of pulsed neutron source like SNS has been built. At same time, only few new reactors have been built in recent years. Most of neutron source have to be closed within less than 20 years.

  The European Spallation Source (ESS) is a project of an intense pulsed neutron source. ESS is a crucial project to maintain the level of expertise of neutron users in Europe and to stimulate a renewal in neutron production, whether through new research reactors or accelerator-driven sources.

  \begin{figure}[!ht]
	\begin{subfigure}[t]{0.5\textwidth}
		\includegraphics[width=\textwidth]{01_Introduction/figures/fig000_NeutronSources_a}
		\caption{Evolution of thermal neutron sources from the neutron discovery to a near future.}
		\label{}
	\end{subfigure}
	~
	\begin{subfigure}[t]{0.5\textwidth}
		\includegraphics[width=\textwidth]{01_Introduction/figures/fig000_NeutronSources_b}
		\caption{Evolution of the European neutron facilities over the next 15 years.}
		\label{}
	\end{subfigure}
	\caption[Status and outlook of neutron sources (from the ESFRI report)]{Status and outlook of neutron sources (from the ESFRI report) \cite{neutron2016}.}
	\label{chap1:fig:NeutronSources}
\end{figure}


  \cleardoublepage
  \section{Bibliography}

  \printbibliography[heading=subbibliography]
\end{refsection}