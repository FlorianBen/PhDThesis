\chapter*{Acknowledgments}
\chaptermark{Acknowledgments}
\addcontentsline{toc}{chapter}{Acknowledgments}
\adjustmtc
First, I would like to thanks all the committee members: Patrick Puzo, Gloria Luzon, Peter Forck, Julien Pancin and Cyrille Thomas for offering their time for the review of this manuscript and the defense. I am also very grateful to the committee members for all their advice and suggestions. I would like to thanks again Peter Forck for his comments during the two different reviews at ESS and for his enthusiasm in beam diagnostics in general; the IPM workshop hosted at GSI was a great opportunity to meet IPM experts from all around the world. I would also like to acknowledge Cyrille Thomas, as well as all his colleagues at ESS, for their implications in this project.

Passons maintenant en Français avec les remerciements des collègues du CEA. L’exercice est plus difficile qu’il n’y paraît puisque pendant cette thèse j’ai été amené à travailler mais aussi échanger avec de nombreuses personnes de l’IRFU et du DEDIP. Je ne peux malheureusement pas faire une liste exhaustive et je m’excuse donc par avance d'éventuels oublis.

Mes remerciements vont d’abord à Eric Delagnes, Philippe Bourgeois et Christine Porcheray qui m’ont accueilli au DEDIP et au 534 pendant ces trois années. Je remercie également Nathalie Le Devedec toujours prête à aider le thésard devant un formulaire cryptique du CEA. Faire une thèse au DEDIP est une expérience vraiment très enrichissante grâce à la grande diversité des projets soutenus par le département.

Je salue tous les étudiants en thèse passés, présents et futurs : Abhilasha, Aurore, David, Geoffrey, Lukas, Loic, Patrick, Marion, Maxence et Morgane. Communiquer avec des personnes aux horizons différents est toujours très enrichissant. Je souhaite bon courage à ceux qui vont commencer ou terminer leurs thèses prochainement.

Merci à Mariam d’avoir partagé son bureau avec moi pendant ces trois ans. Je tiens également à saluer les collègues du DEDIP nord avec qui j’avais l’habitude de déjeuner : Benjamin, Fanny, Hector, Ioannis (Giomataris), Ioannis (Katsioulas), Laura, Paco, Patrick, Thomas, Xavier-François. Les repas de midi sont l’occasion unique pour échanger sur des sujets variés (toujours avec sérieux). C’est aussi à ce moment que l’on peut résoudre pas mal de problèmes techniques en discutant des expériences de chacun.

Je remercie Olivier, Geoffrey et Daniel du DAP avec qui j’ai pu profiter de la conférence ANNIMA même si j’étais dans un des moments les plus sombres de la rédaction (un mois et demi avant la date limite). Au passage, merci à Olivier et à tous ceux qui ont participé aux répétitions de la soutenance pour leurs conseils et remarques.

David, j’espère que tu mèneras à bien tous tes projets professionnels et sportifs. Courage à toi mais aussi à Fabien, Maxence et Alain afin de promouvoir et gérer les PC de simulations. Je salue également les autres coureurs du 534 et du 141: Julie, Etienne et Fabrice. Même si j’étais beaucoup moins sérieux et motivé que vous, j’ai pu maintenir un niveau de forme suffisant lors de ces trois années.

Je souhaite bonne continuation à tous les autres collègues du DEDIP, nordistes et sudistes (merci Xavier pour les photos), de ATLAS à ScanPyramid (je n’ai pas trouvé de projet avec la lettre Z).

Maintenant, mes remerciements vont se concentrer sur les différents acteurs de l’IRFU qui ont collaboré sur ce projet. En premier lieu, Florence Ardellier, Pierre Bosland et Christophe Mayri pour nous avoir considérablement aidé sur les nombreux aspects que comporte le projet ESS. Les mois qui arrivent vont sans doute voir venir la concrétisation de ces dernières années de travail.

Ensuite, l'ensemble des équipes du DACM et en particulier le laboratoire du LEDA et les pilotes faisceau et RF de IPHI qui ont dû nous supporter pendant les quelques mois de tests faisceau. Je vous souhaite plein de réussites dans les ambitieux projets qui arrivent.

Partons maintenant au DIS côté “informatique” : merci à Yannick et Victor qui nous ont grandement aidé à avoir un contrôle système sérieux, ce qui nous a facilité la tâche lors de nos tests en faisceau. Puis au DIS côté “mécanique” avec Loris et les dessinateurs pour leurs idées astucieuses qui ont fait en sorte que les détecteurs soient fonctionnels (et esthétiques, le plus important m’a-t-on dit).

Retournons au DEDIP, d’abord Philippe avec qui j’ai pu apprendre beaucoup sur l'électronique analogique, je suis sûr que tu profites pleinement de ta retraite en Touraine. Jean-Philippe, pour ses idées qui nous ont aidées lors des situations d’urgence. Gérard qui a récemment rejoint notre projet garanti sans matière explosive. Caroline qui m’a aidé à monter les détecteurs dans la fournaise d’un labo sans climatisation pendant un mois de juillet caniculaire. Et Francesca avec qui j'ai appris énormément de par son expérience dans les détecteurs et la simulation mais aussi par sa rigueur scientifique et linguistique qui m’a beaucoup aidé lors de la rédaction.

Enfin mes remerciements vont s'adresser aux deux personnes sans qui cette thèse n'aurait pas pu être possible. D'abord, Esther, ma directrice de thèse, qui a toujours veillé au bon déroulement de la thèse et a su me remotiver pendant les moments les plus délicats. Merci également pour tes suggestions qui ont grandement améliorée la compréhension de la thèse et m'avoir aidé à affronter la jungle ADUM pendant la phase de rédaction. Je te souhaite de la réussite dans les différents projets que tu soutiens.

Ensuite, Jacques, mon responsable de thèse, qui m’a fait confiance sur ce projet et qui m’a apporté tout son savoir pour me permettre de mener à bien mon travail. J’espère que tes différents projets diagnostiques accélérateurs vont se finaliser et que tu continueras à t’amuser à la tête du LASYD. J’espère avoir l’occasion d’aller observer un jour le ciel étoilé du Cantal que tu m’as tant vanté. 

Je pense qu’il est difficile d’évaluer la chance que j’ai eu d’avoir un encadrement de cette qualité avec autant de bienveillance. 

Pour finir, je souhaite remercier mes parents et ma famille pour leur soutien dans les moments de tous les instants les plus difficiles de la thèse ce qui m’a permis de toujours continuer.