\begin{figure}[!ht]
  \begin{subfigure}[t]{0.5\textwidth}
    \includesvg[width=\textwidth]{04_IPHI_Test/figures/fig000_vacuum_rga_a}
    \caption[An RGA spectrum dominated by water.]{An RGA spectrum dominated by water (high peaks at $18$ $17$). Hydrogen (peak at $2$) and nitrogen (peak at $28$) are also present, as well as some heavy carbonate species (peaks above $>28$).}
    \label{chap4:vacuum_rga_b}
  \end{subfigure}
  ~
  \begin{subfigure}[t]{0.5\textwidth}
    \includesvg[width=\textwidth]{04_IPHI_Test/figures/fig000_vacuum_rga_b}
    \caption{After few days of pumping or baking, the water peaks are reduced and the hydrogen becomes the main compound.}
    \label{chap4:vacuum_rga_a}
  \end{subfigure}
  \caption[Two types of RGA spectra recorded during the tests]{Two types of RGA spectra recorded during the tests. Measuring the vacuum composition is mandatory to correctly extrapolate the signal.}
  \label{chap4:vacuum_rga}
\end{figure}
