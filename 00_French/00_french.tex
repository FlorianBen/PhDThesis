\chapter*{Résumé en Français}
\addcontentsline{toc}{chapter}{Résumé en Français}%

% ESS Global
Cependant produire des neutrons est beaucoup plus compliqué que de produire des particules chargées ou même des photons. Une des méthodes pour produire des neutrons consiste à utiliser des sources composites type alpha-béryllium. Cependant ces sources sont limitées à des faibles flux de neutrons. Les infrastructures de recherche actuelles reposent plutôt sur des réacteurs à fission. Dans un réacteur à fission l’Uranium 235 sous l’impact d’un neutron se sépare en deux noyaux plus légers avec une émission de neutrons. Certains de ces neutrons vont créer une réaction en chaîne avec d’autres atomes d’Uranium 235. On estime qu’en moyenne 2.5 neutrons sont libérés par atome d’Uranium 235 fissé. Le coût de fonctionnement de ce type d’installation est important car la gestion de la sécurité et des matières radioactives est obligatoire et coûteuse. Les réacteurs de recherche souffrent également de l’image du nucléaire ce qui peut freiner l’investissement dans ce type d’installation.

Depuis quelque années les sources de neutrons par spallation sont devenues une alternative crédible face aux réacteurs et de nombreux projets de source à spallation ont vu le jour. Cela est rendu possible grâce aux progrès réalisés dans les technologies des accélérateurs de particules. Le fonctionnement de ce type de source est le suivant: des protons de hautes énergies (de la centaine de MeV au GeV) vont entrer en collision avec une cible solide type métal. Les noyaux de la cible vont se désintégrer sous la violence du choc et des neutrons sont libérés sur un spectre très large. Les neutrons peuvent être ensuite modérés et transportés vers différentes expériences.

La Source de Spallation Européen (ESS) sera la future source de neutrons la plus brillante. Sa construction à commencer à Lund, en Suède. Les premiers neutrons sont attendus pour 2022 afin de prévenir la fermeture des principaux réacteurs et au LLB à l’ILL d’ici les prochaines années. Lors du démarrage du faisceau, 15 instruments de spectroscopie, de réflectométrie et de diffraction neutronique pour des application varié seront disponibles pour les chercheurs et les partenaires industriels. Puis dans une phase d’extension 7 nouveaux instruments seront installés à ESS.
