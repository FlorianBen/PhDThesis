
%%% a lifehuck to adgust the font size and spacing %%%
\makeatletter
\newcommand*\mysize{%
  \@setfontsize\mysize{9.5}{9.0}%
}
\makeatother

\chapter*{}
\addcontentsline{toc}{chapter}{Abstract}%
%\thispagestyle{empty}
\tikzset{external/export next=false}
\begin{tikzpicture}[remember picture, overlay]
  %%% the University+ED logo %%%
  \node [anchor=north west, shift={(1.2 cm,-0.2cm)}] at (current page.north west) {\includegraphics[width=7.5cm]{00_Guards/Logos/pheniics.png}};
  \mysize 
  \node [anchor=north, yshift=-3.1 cm, text width=18cm, inner sep=.3cm] (resume) at (current page.north) {
    %\begin{minipage}{\linewidth}
    \textbf{Title:} Design of non-invasive profile monitors for ESS proton beam\\
    %%% key words %%%
    \textbf{Key words:} \textit{Proton accelerator, Beam diagnostics, Instrumentation, Detector} \\
    \textbf{Abstract:} The European Spallation Source (ESS) will be a research infrastructure dedicated to sciences using neutrons as probes. The source is currently under construction in Lund, Sweden, and will be the world’s brightest pulsed source of neutrons. As its name suggests, the production of neutrons is ensured by the spallation process: high energy protons will impinge a tungsten target. To accelerate the protons, a powerful 2 GeV linear accelerator is being built. The accelerator can be split in two parts. A “hot” part is responsible for acceleration up to 90 MeV. Then a “cold” part made of superconducting cavities cooled with liquid helium is used to reach the highest energies. The high intensity of 62.5 mA and he long pulse of 2.86 ms  repeated 14 times per second, lead to an incredible beam power of 5 MW in average and 125 MW in peak. The knowledge of the beam is therefore mandatory to ensure the commissioning, i.e. the beam tuning in order to achieve a proper and safe functioning of the machine. Different diagnostics will be installed along the accelerator to fulfil these tasks. 
    This thesis deals with the development of a non-invasive transverse profiler for the cold part of the ESS accelerator: the Ionization Profile Monitor (IPM).The thesis focuses on critical aspects of the IPMs to guarantee its feasibility in ESS beam conditions. These monitors are based on the ionization of the residual gas induced by the proton beam inside the beam pipe. A transverse electrical field is generated between both parallel plates of the IPM. The electrons or ions drift, with respect to the electric field, towards a segmented detector allowing the reconstruction of the beam profile in one transverse direction. For a complete transverse profile, it is necessary to add a second profiler tilted by 90°.
    Several challenges for facing IPM to the ESS conditions, which may compromise their use, are described:
    \begin{itemize}
      \item The weak counting rates due to the low ionization cross-sections at high energy (90 to 2000 MeV) and to the low residual gas pressure of 10-9 mbar.
      \item The electric field homogeneity inside the IPM, which is relevant for insuring a precise profile measurement, was not obvious in the narrow vacuum chambers devoted to them.
      \item The large Space Charge Effect of the beam, distorting the measured profile by deviating the ionization by-product trajectories. This fundamental aspect may comprise compromise the use of an IPM for beam profile measurements.
    \end{itemize}
    Once these former studies done, we selected the three reliable read-out systems based on:
    \begin{itemize}
      \item Conductive strips read by a multichannel charge integrator.
      \item Micro-Channel Plates coupled with phosphor screen (pMCP).
      \item A silicon detector developed at CERN and foreseen for the future PS beam profiler.
    \end{itemize}
    This work was the object of the Preliminary Design Review (PDR 2017/01) marking the beginning of the construction phase of the different prototypes.
    Preliminary tests discarded the possibility of using silicon detectors due to the low ion energies. 
    Starting from scratch, IPMs, reference monitors and a test bench were designed and installed at the IPHI proton accelerator at Saclay. Close ESS conditions were achieved to validate an IPM solution and our simulations. 
    The test campaigns showed that an MCP is mandatory to detect signal. Moreover, the optical IPM (pMCP + Camera) is the preferred solution since it provides higher sensitivity. Feedbacks from the prototype test campaigns, allows us to deliver an IPM final design presented during the Critical Design Review (CDR 2019/02) leading to the beginning of the production phase.
  };
  
  \node [anchor=north, yshift=-0.3 cm, text width=18cm, inner sep=.3cm] (abstract) at (resume.south) {
    \textbf{Titre :} Conception de profileurs non intrusifs pour le faisceau de protons de ESS\\
    \textbf{Mot clés :} \textit{Accélérateur de proton, Diagnostic faisceau, Instrumentation, Détecteur} \\
    \textbf{Résumé :} La source européenne de spallation (ESS) sera une infrastructure de recherche dévolue aux sciences  utilisant les neutrons comme sonde d’observation. Elle est actuellement en construction à Lund, en Suède, et sera la plus brillante des sources de neutrons pulsées au monde. Comme son nom l'indique, la production des neutrons est assurée par les processus de spallation : des protons à haute énergie bombardant une cible de tungstène. Le faisceau de protons est généré par un puissant accélérateur linéaire de 2 GeV qui peut être divisé en deux parties : une partie "chaude" qui accélère les protons jusqu'à 90 MeV, suivie d’une partie « froide » constituée de cavités supraconductrices refroidies à l'hélium liquide, permettant d’atteindre les 2 GeV. La forte intensité de 62.5 mA et la longue impulsion de 2,86 ms répétée 14 fois par seconde, conduisent à une puissance moyenne de faisceau de 5 MW et une puissance crête de 125 MW. La connaissance du faisceau est donc indispensable pour la mise en service, c'est-à-dire le réglage du faisceau afin d'assurer un fonctionnement correct et sûr de la machine. Différents diagnostics seront installés le long de l'accélérateur pour remplir ces tâches.
    Cette thèse traite du développement d'un profileur transverse non invasif pour la partie froide de l’accélérateur de ESS : les Ionization Profile Monitors (IPM). La thèse se concentre sur les aspects critiques des IPM afin de s’assurer de leur faisabilité dans les conditions du faisceau de ESS. Ces moniteurs sont basés sur l’ionisation induite par le passage des protons du gaz résiduel présent dans le tube de l’accélérateur. Un champ électrique est appliqué entre deux plaques parallèles de l'IPM. Les électrons ou les ions dérivent vers un détecteur segmenté permettant de reconstruire le profil dans une direction transverse du faisceau.
    Plusieurs défis, qui auraient pu compromettre l’utilisation des IPM pour les mesures des profils de faisceau à ESS, sont décrits :
    \begin{itemize}
      \item Les faibles taux de comptage dus aux faibles sections efficaces d'ionisation à haute énergie (90 à 2000 MeV) ainsi qu’aux basses pressions du gaz résiduel de l’ordre de 10-9 mbar.
      \item L'homogénéité du champ électrique à l'intérieur de l'IPM, essentiel pour assurer des mesures de profils précises mais difficile pour les chambres à vide étriquées des IPM.
      \item L’importante charge d'espace du faisceau, qui distord le profil mesuré en déviant les
      trajectoires des produits d'ionisation. Cet aspect fondamental peut remettre en cause l’utilisation d’IPM pour faire des mesures fiables de profil de faisceau.
    \end{itemize}
    Une fois ces études terminées, nous avons sélectionné trois systèmes de lecture fiables, basés sur :
    \begin{itemize}
      \item Des pistes conductrices lues par un intégrateur de charge multicanal.
      \item Des détecteurs à micro-canaux couplés à un écran phosphore (pMCP).
      \item Un détecteur de silicium développé au CERN, et utilisé en particulier pour le futur profileur du faisceau du PS.
    \end{itemize}
    Ces études ont fait l’objet d’une Revue de Conception Préliminaire (PDR 2017/01) marquant le début de la construction des différents prototypes.
    Les tests préliminaires ont écarté la possibilité d'utiliser des détecteurs au silicium en raison des trop faibles énergies des ions incidents.
    En partant de zéro, des IPM, des moniteurs de référence et un banc d’essai ont été conçus et installés sur l’accélérateur de protons IPHI à Saclay. Les conditions expérimentales de ESS ont été reproduites afin de valider une solution pour les IPM, ainsi que tester nos modèles.
    Les campagnes de test ont montré qu'un MCP était nécessaire pour détecter le signal d’ionisation. De plus, l'IPM optique (pMCP + caméra) est la solution recommandée car elle offre une sensibilité plus élevée. Le retour d’expérience accumulé lors des tests des prototypes, nous a permis de proposer une conception quasi finale d’un IPM, présentée lors de la Revue Critique de Conception (CDR 2019/02), menant au début de la phase de production.
  };
  
  %%% draw a purple frame around each abstract %%%
  \draw[line width=1 pt, violet!80!red] (resume.south west) -- (resume.north west) -- (resume.north east) -- (resume.south east) -- (resume.south west);
  \draw[line width=1 pt, violet!80!red] (abstract.south west) -- (abstract.north west) -- (abstract.north east) -- (abstract.south east) -- (abstract.south west);
  
  %%% footnote %%%
  \node [anchor=south west, violet!80!red, shift={(1.2 cm,0.5cm)}, inner sep=0.2pt] at (current page.south west) {
    \begin{minipage}{12cm}
      {\bf Universit\'{e} Paris-Saclay\\Espace Technologique / Immeuble Discovery\\Route de l’Orme aux Merisiers RD 128 / 91190 Saint-Aubin, France} \\
    \end{minipage}
  };
  
  %%% the "e" image at the bottom %%%
  \node [anchor=south east, violet!80!red!80!black, shift={(-1.5 cm,0.5cm)}, inner sep=0pt] at (current page.south east) {\includegraphics[width=1.6 cm]{00_Guards/Logos/e.png}};	
\end{tikzpicture}