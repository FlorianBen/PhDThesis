
%%% a lifehuck to adgust the font size and spacing %%%
\makeatletter
\newcommand*\mysize{%
	\@setfontsize\mysize{9.5}{9.0}%
}
\makeatother

\chapter*{}
\addcontentsline{toc}{chapter}{Abstract}%
%\thispagestyle{empty}
\tikzset{external/export next=false}
\begin{tikzpicture}[remember picture, overlay]
	%%% the University+ED logo %%%
	\node [anchor=north west, shift={(1.2 cm,-0.2cm)}] at (current page.north west) {\includegraphics[width=7.5cm]{00_Guards/Logos/pheniics.png}};
	\mysize 
	\node [anchor=north, yshift=-3.1 cm, text width=18cm, inner sep=.3cm] (resume) at (current page.north) {
		%\begin{minipage}{\linewidth}
		\textbf{Title:} Design of non-invasive profile monitors for ESS proton beam\\
		%%% key words %%%
		\textbf{Key words:} \textit{Proton accelerator, Beam diagnostics, Instrumentation, Detector} \\
		\textbf{Abstract:} The European Spallation Source or ESS will be a European research infrastructure dedicated to neutronic sciences. The source is currently under construction in Lund, Sweden, and will be the world's brightest pulsed source of neutrons. As its name suggests, the production of neutrons is ensured by the spallation process: high energy protons will impinge a tungsten target. To accelerate the protons a powerful $2\,\mathrm{GeV}$ linear accelerator is being built. The accelerator can be schematized in two parts. A hot part is responsible for acceleration up to $90\,\mathrm{MeV}$. Then a cold part using superconducting cavities cooled with liquid helium is used to reach the highest energies. The long pulse of $2.86\,\mathrm{ms}$ and the high intensity of $62.5\,\mathrm{mA}$ repeated $14$ times per second lead to an incredible beam power of $5\,\mathrm{MW}$ in average and $125\,\mathrm{MW}$ in peak. Knowledge of the beam is therefore mandatory to ensure the commissioning, tuning and proper functioning of the machine. Several diagnostics will be installed along the accelerator to ensure these tasks. CEA/IRFU is actively involved in the ESS project mainly at the accelerator level and its diagnostics.
		
		This thesis deals with the development of a non-invasive beam profile for the cold part of the ESS accelerator and focuses on critical aspect of the detectors to guaranteed its feasibility at ESS. The detection is based on the continuous ionization of the residual gas by the proton beam. An electrical field is generated between two parallel plates. The electrons or ions, following the polarity, are drawn toward a segmented detector so the profile is reconstructed in one direction. For a complete transverse profile it is necessary to place a second profiler afterwards in the other direction. The primary count rates will determine the detection technology. Three different technologies have been selected. The first one is based on conductive strips read by a multi-channel charge integrator. The second technology uses micro-channel plates coupled with a phosphor screen. Finally the last solution uses a silicon detector developed at CERN and used for the future PS beam profiler. 
		
		A second fundamental aspect to study concerns the distortions of the beam; caused by the non-uniformity of the electric field, the initial movement of the particles and the electromagnetic field induced by the beam. The simulations show that the profile can be correctly reconstructed if ions are used.
		
		All of these studies were presented during a Preliminary Design Review which marked the beginning of the construction phase of the different prototypes. Preliminary tests discarded the possibility of using silicon detectors with low energetic ions. Finally, beam measurements were carried out during two test campaigns at IPHI, a $3\,\mathrm{MeV}$ linear proton accelerator at CEA Saclay. Both types of detectors worked well and the results has been compared with other diagnostics. The tests also permit to check the characteristics of the profilers under conditions close to ESS ones. The results of the prototypes were presented during the Critical Design Review. The results of the prototypes were presented during the Critical Design Review leading to the beginning of the production phase.
		
	};
	
	\node [anchor=north, yshift=-0.3 cm, text width=18cm, inner sep=.3cm] (abstract) at (resume.south) {
		\textbf{Titre :} Conception de profileurs non intrusifs pour le faisceau de protons de ESS\\
		\textbf{Mot clés :} \textit{Accélérateur de proton, Diagnostic faisceau, Instrumentation, Détecteur} \\
		\textbf{Résumé :} La Source de Spallation Européenne ou ESS sera une infrastructure de recherche Européenne dédiée à la science neutronique appliquée. La source est actuellement en cours de construction à Lund, en Suède et sera la plus brillante des sources pulsées de neutrons au monde. Comme son nom l’indique la production de neutron est assurée par le phénomène de spallation : des protons de hautes énergies vont entrer en collision avec une cible en tungstène. Pour accélérer les protons un puissant accélérateur linéaire de $2\,\mathrm{GeV}$ est en cours de construction. La particularité de cet accélérateur est sa longue impulsion de 2.86 ms et son intensité maximale de 62.5 mA répétées $14$ fois par seconde, soit $5\,\mathrm{MW}$ de puissance moyenne et $125\,\mathrm{MW}$ pic. L’accélérateur peut se scinder en deux parties. Une partie chaude se charge de l’accélération jusqu’à $90\,\mathrm{MeV}$. Puis une partie froide utilisant des cavités supraconductrices refroidies à l’hélium liquide est utilisée pour atteindre les plus hautes énergies. Une connaissance et une maîtrise de l’accélérateur est donc obligatoire pour garantir la sécurité, la mise en service et le bon fonctionnement de l’installation. Des diagnostics seront installés tout le long de l’accélérateur pour assurer ces tâches. L'IRFU est activement impliqué dans le projet ESS principalement au niveau de l’accélérateur et de ses diagnostics.
		
		Cette thèse porte sur le développement d’un mesureur de profil faisceau non invasif pour la partie froide de l’accélérateur de ESS et traite l’ensemble des problématiques associées. La détection se base sur l’ionisation continue du gaz résiduel par le faisceau de protons. Le profileur consiste en deux plaques parallèles qui vont extraire les électrons ou les ions, suivant la polarité, et les envoyés sur un détecteur segmenté. Le profile est donc reconstruit dans une direction. Pour un profil transverse complet il est nécessaire de placer un second profileur à la suite dans l’autre direction.
		Des études sur différents aspects du détecteur ont été effectuées afin de monter sa faisabilité à ESS. En particulier les taux de comptage primaires qui vont déterminer la technologie de détection utilisée. Trois technologies ont été sélectionnées. La première technologie est basée sur des pistes conductrices lues par un intégrateur de charge multivoies. La seconde technologie utilise des galettes micro-canaux (MCP) avec écran phosphore lue par une caméra. Enfin la dernière solution utilise un détecteur silicium développé et utilisé par le CERN pour le futur profileur du PS. Le second aspect fondamental à étudier concerne les distorsions du faisceau. Elles peuvent être causées par les non-uniformités du champ électrique, le mouvement initial des particules et le champ électromagnétique induit par le faisceau. Les simulations montrent que le profil peut être correctement reconstruit à condition de collecter les ions.
		
		L’ensemble de ces études ont été présentées lors d’une revue préliminaire de projet qui a marqué le début de la phase de construction des différents prototypes. Des tests préliminaires ont écarté la possibilité d’utiliser les détecteurs silicium avec des ions de basses énergies. Enfin des mesures en faisceau ont été réalisés durant deux campagnes de tests à IPHI, un accélérateur de proton linéaire de $3\,\mathrm{MeV}$ au CEA Saclay. Les deux types de détecteur ont correctement fonctionné et les résultats ont pu être comparés avec d’autres diagnostics. Les tests ont également permis de vérifier les caractéristiques des profileurs dans des conditions permettant l’extrapolation aux conditions ESS. La technologie éprouvée des MCP reste la plus avantageuse. Les résultats des prototypes ont été présentées lors de la revue critique et à l’heure actuelle les détecteurs finaux sont en cours en production pour une première livraison en début d’année prochaine (2020).
	};
	
	%%% draw a purple frame around each abstract %%%
	\draw[line width=1 pt, violet!80!red] (resume.south west) -- (resume.north west) -- (resume.north east) -- (resume.south east) -- (resume.south west);
	\draw[line width=1 pt, violet!80!red] (abstract.south west) -- (abstract.north west) -- (abstract.north east) -- (abstract.south east) -- (abstract.south west);
	
	%%% footnote %%%
	\node [anchor=south west, violet!80!red, shift={(1.2 cm,0.5cm)}, inner sep=0.2pt] at (current page.south west) {
		\begin{minipage}{12cm}
			{\bf Universit\'{e} Paris-Saclay\\Espace Technologique / Immeuble Discovery\\Route de l’Orme aux Merisiers RD 128 / 91190 Saint-Aubin, France} \\
		\end{minipage}
	};
	
	%%% the "e" image at the bottom %%%
	\node [anchor=south east, violet!80!red!80!black, shift={(-1.5 cm,0.5cm)}, inner sep=0pt] at (current page.south east) {\includegraphics[width=1.6 cm]{00_Guards/Logos/e.png}};	
\end{tikzpicture}